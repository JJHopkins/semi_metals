p1
assume:We assume that the two anisotropic half-spaces are composed of anisotropic cylinders of
radii R 1 and R 2 at volume fractions v 1 and v 2 , with c 1,⊥ ( c 2,⊥ ) and c 1, ( c 2, ) as the transverse and longitudinal
dielectric response functions of the cylinder materials 

p2. DERIVATION: 
assume:In order to get the interaction free energy between two anisotropic cylinders we assume that both semi-infinite
substrates (half-spaces),

p2 assume:
We assume in what follows that all the response functions are bounded and finite

p2. We use the Pitaevskii ansatz in order to extract the interactions between two infinite anisotropic cylinders at
all separations and angles from the interaction between two semi-infinite half-spaces of anisotropic uniaxial dielectric
material. We start with the fully retarded van der Waals - dispersion interactions between two semiinfinite anisotropic
dielectric slabs [6]. The full interaction form is quite involved, but it has a simple limit if the two semiinfinite slabs,
L and R, separated by an isotropic medium of thickness , are composed of rarefied material.

p3 assume:
Here R 1 and R 2 are the cylinder radii, assumed to be the smallest lengths in the problem [11]


P3 SKEWED CYL- FULLY RET:
This is the final result for the cylinder-cylinder interaction at all angles when the radii of the cylinders are the smallest
lengths in the system. It includes retardation and the full angular dependence. Some simple limits can be obtained
form this general expression.

p5. bc G for SR case goes to inf for theta go to 0. where L is the length of the cylinder. In this way one sees that in the limit θ −→ 0 and for /L

p6 SKEW LOW TEMP
We now rework this equation to obtain the retarded result for the interaction between two semiconducting cylinders.
Note here that we can not derive the Casimir limit properly as our formulation is not valid for nominally infinite
zero-frequency
(Drude-like) dielectric response. For that case see Ref. 11.

p6 PARLLEL RETARDED
account: Taking this [Abel transform] into account when considering Eqs. 30, we remain with

p8 PARALLEL RETARDED
approx: In na exact theory, for finite length cylinders, A( ), Eq. 48, and A (0) ( ) + A (2) ( ), Eqs. 14 and 16, should coincide
exactly. Our calculation is approximate and off hand we would not expect them to coincide. So we have to compare
the Hamaker coefficient for the parallel case

p9 P ZERO TEMP LIMIT
V.
PARALLEL CYLINDERS - ZERO TEMPERATURE RESULT
As with skewed cylinders, we can take the zero temperature limit where the summation
√ over the Matsubara
frequencies becomes an integral over n with dn = /(2π k B T )dω. Again we introduce x = c t 2 + 1 ω. Then, as for
skewed cylinders, we obtain the interaction free energy per unit length of two parallel cylinders,
The spatial dependence is, again, one power higher in the retarded regime than in the non-retarded regime. All the
frequency dependence of the material properties in the retarded limit is again reduced to the static response as in the
Lifshitz analysis [14].

p10 SCREENED ZERO FREQ TERM
p10 approx, account, assume: 
Because of the presence of salt the zero frequency (classical) contribution to the Hamaker coefficients is screened.
This means that instead of the Laplace equation one should take into 
account the linearized Debye-Huckel equation.
This is of course 
approximate and more sophisticated statistical mechanical theories could be taken into account.
Nevertheless we remain within the framework of the inearized Debye-Huckel theory. One should also take into
account that the free charges are present in the medium (electrolyte solution) as well as within the interacting bodies

assumed to be bad conductors (see L.P. Pitaevskii, PRL 101, 163202 (2008)), i.e., the number of charge carriers
is small and it obeys the Boltzmann statistics. This furthermore implies that the electrostatic potential obeys the
Debye-H ̈
uckel equation of the form

p10 assume:
We
assume in what follows that all the response functions are bounded and finite. Furthermore the mobile charges occupy
the medium m as well as both L and R half-spaces and (see R. Podgornik and V.A. Parsegian, PRL 80 1560 (1998))
because solutions of the wave equation require continuity of ∇φ(r) perpendicular to both L and R interfaces, it is
the perpendicular dielectric response that determines the ionic screening lengths in both semiinfinite regions, i.e.,

p10 assume:
where we assumed that
n L,R = n c v + n m (1 − v) = n m + v(n c − n m )

p11 
account:To take account of anisotropy, ψ is for angular integration
over all directions in radial wave vectors Q. The functions ∆ L,m (ψ) and ∆ R,m (θ − ψ) can be obtained in the standard
way (see R. Podgornik and V.A. Parsegian, PRL 80 1560 (1998)) as

account: The derivatives w.r.t. v should then take into account all the v dependencies as indicated by the argument.

approx: Instead of going through the derivation once again for this case we note that within the DH approximation formally
ω 2
Eq. 7 would remain the same if we make the substitution m c 2 n −→ κ 2 , where κ 2 is the inverse Debye screening
length. Thus we obtain (see R. Podgornik and V.A. Parsegian, PRL 80 1560 (1998))

p11 This general result has various interesting limits. In the case of vanishing salt concentration in the medium, that
is κ m −→ 0 and with F(θ) = 2 cos 2 θ + 1 /2 7 we end up with the skewed
configuration....In the opposite case of strong ionic screening

p13 INTERACTIONS AT SMALL SPACINGS
We take the main results from Ref. [21]. In the limit of small separations between the two cylinders /a −→ 0, we
reformulate the approach based on the Derjaguin method and introduced for a
single cylinder and a substrate.
p14This is the final expression for the interaction free energy between two cylinders at a general angle θ and separation
in the proximal limit.

p14 GOSBERGS INTERP FORMULA (INTERACTIONS AT SMALL SPACINGS)
Our formulas are however not of the Hamaker-type that Grosberg was using, so the Hamaker coefficient for the small
and large separation limit are not the same.

p15 NON-PAIRWISE ADDITIVE EFFECTS
account: To assess the non-pairwise additive effects in arrays of cylindrical molecules I first formulate a model to take them
into account. 

assume: I assume a central anisotropic cylinder of radius R 1 surrounded by a concentric cylindrical surface at
radius R 2 of anisotropic medium that is composed of anisotropic cylinders at an area packing density
√ of N and thus
of volume fraction v = N πR 1 2 . As the apothem of a hexagon is a = 2 1 (R 1 + R 2 ) and its area A = 2 3a 2 I obtain for
the volume fraction (equal to area fraction) of hexagonal packing


p11 SCREENED ZERO FREQ TERM

p17 account 
I note here that the product of a Bessel’s function with its derivative has to be expanded to the first subdominant
order by taking into account that

p17WhatI will first investigate is the asymptotic limit of lim R 2 −→ ∞.
